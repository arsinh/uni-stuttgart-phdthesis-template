% ======================================================================
% Glossary
% ======================================================================

% \newcommand with variable command names
\newcommand*{\newnamecommand}{\@star@or@long\new@name@command}
\newcommand*{\new@name@command}[1]{\expandafter\new@command\csname #1\endcsname}

\iftoggle{partialCompileMode}{
  % define dummy commands that work as if partialCompileMode was disabled
  \newcommand*{\newnotation}[3]{}
  \newcommand*{\newnotationcommand}[6][0]{\newcommand*{#2}[#1]{#3}}
  \newcommand*{\newnotationcommandoptarg}[7][]{\newcommand*{#3}[#2][#1]{#4}}
  \newcommand*{\makecommandnotation}[4]{}
  \newcommand*{\usenotation}[1]{\ignorespaces}
  
  \newcommand*{\newgacronym}[3][]{%
    \if\relax\detokenize{#1}\relax%
      \newnamecommand{#2}{\MakeUppercase{#2}\xspace}%
      \newnamecommand{#2s}{\MakeUppercase{#2}s\xspace}%
      \newnamecommand{#2c}{\MakeUppercase{#2}\xspace}%
      \newnamecommand{#2sc}{\MakeUppercase{#2}s\xspace}%
    \else%
      \newnamecommand{#2}{#1\xspace}%
      \newnamecommand{#2s}{#1s\xspace}%
      \newnamecommand{#2c}{\makefirstuc{#1}\xspace}%
      \newnamecommand{#2sc}{\makefirstuc{#1}s\xspace}%
    \fi%
  }
  \newcommand*{\hidegacronym}[1]{}
}{
  % define new glossary style based on long (uses longtable)
  % in which the column widths are fixed and space before/after the
  % table is removed
  \newglossarystyle{longraggedfixed}{
    % base glossary style
    \setglossarystyle{long}
    
    % custom flag to check if we're in the glossary
    \newif\ifinglossary
    
    \renewenvironment{theglossary}%
    {%
      % remove space before/after glossary
      \setlength{\LTpre}{0pt}%
      \setlength{\LTpost}{0pt}%
      %
      % set custom flag that we are now in the glossary
      \inglossarytrue%
      %
      % set font size
      \small%
      %
      % use single spacing
      \begin{spacing}{1}%
        % table header
        \begin{longtable}{@{}L{0.15\textwidth}@{}L{0.85\textwidth}@{}}%
          \textbf{Symbol}\vspace{2mm}&%
          \textbf{Meaning\hfill{}Page with First Occurrence}\vspace{2mm}%
          \endhead%
    }%
    {%
      % table footer
        \end{longtable}%
      \end{spacing}%
      % set custom flag that we are now outside the glossary
      \inglossaryfalse%
    }
  
    % add dotted leaders between entry description and page number
    % (called for every row of the table)
    \renewcommand*{\glossentry}[2]{%
      \glsentryitem{##1}\glstarget{##1}{\glossentryname{##1}}&%
      \glossentrydesc{##1}\leavevmode\kern3pt\leaders\hbox{%
        \hspace{0.2237em}.\hspace{0.2237em}%
      }\hfill\kern0pt%
      ##2%
      \tabularnewline%
    }
  }
  
  % use new glossary style
  \setglossarystyle{longraggedfixed}
  
  % add field to glossary entries that stores whether the entry
  % has been used at least once in the document
  \glsaddstoragekey{myused}{false}{\glsmyused}
  \glsaddstoragekey{myisacronym}{false}{\glsmyisacronym}
  
  % \newnotation{sortKey}{entrySymbol}{entryText}
  % adds notation to the glossary without command,
  % only to be used with \usenotation
  \newcommand*{\newnotation}[3]{%
    \newglossaryentry{#1}{text={},sort={#1},name={#2},description={#3}}%
  }
  
  % \newnotationcommand[numberOfArgs]{\commandName}
  %   {commandDefinition}{sortKey}{entrySymbol}{entryText}
  % adds notation to the glossary and creates a command
  \newcommand*{\newnotationcommand}[6][0]{%
    \newcommand*{#2}[#1]{#3}%
    \makecommandnotation{#2}{#4}{#5}{#6}%
  }
  
  % \newnotationcommandoptarg[defaultValue]{numberOfArgs}{\commandName}
  %   {commandDefinition}{sortKey}{entrySymbol}{entryText}
  % is like \newnotationcommand, but with an optional argument for
  % the command to be created
  \newcommand*{\newnotationcommandoptarg}[7][]{%
    \newcommand*{#3}[#2][#1]{#4}%
    \makecommandnotation{#3}{#5}{#6}{#7}%
  }
  
  % \makecommandnotation{\commandName}{sortKey}{entrySymbol}{entryText}
  % adds notation to the glossary and modifies \commandName such that
  % the glossary entry is referenced on every use of \commandName
  % (except if used in the glossary itself)
  \newcommand*{\makecommandnotation}[4]{%
    \newglossaryentry{#2}{text={},sort={#2},name={#3},description={#4}}%
    % only reference entry if outside glossary
    % (otherwise, the page number of the glossary would be displayed)
    \pretocmd{#1}{\ifinglossary\else\usenotation{#2}\fi}{}{}%
  }
  
  % \usenotation{sortKey}
  % explicitly "uses" the notation given by sortKey without printing anything
  \newcommand*{\usenotation}[1]{%
    \glsfieldgdef{#1}{myused}{true}%
    \glsdisp{#1}{\relax}\ignorespaces%
  }
  
  % shortcut for \newacronym, automatically generating
  % a corresponding shortcut for \gls
  \newcommand*{\newgacronym}[3][]{%
    \if\relax\detokenize{#1}\relax%
      \newacronym[sort={Ü#2},description={\makefirstuc{#3}}]%
      {Ü#2}{\MakeUppercase{#2}}{#3}%
    \else%
      \newacronym[sort={Ü#2},description={\makefirstuc{#3}}]{Ü#2}{#1}{#3}%
    \fi%
    \glsfieldgdef{Ü#2}{myisacronym}{true}%
    \newnamecommand{#2}{\glsfieldgdef{Ü#2}{myused}{true}\gls{Ü#2}\xspace}%
    \newnamecommand{#2s}{\glsfieldgdef{Ü#2}{myused}{true}\glspl{Ü#2}\xspace}%
    \newnamecommand{#2c}{\glsfieldgdef{Ü#2}{myused}{true}\Gls{Ü#2}\xspace}%
    \newnamecommand{#2sc}{\glsfieldgdef{Ü#2}{myused}{true}\Glspl{Ü#2}\xspace}%
  }
  
  % hide specific acronyms (that are only used once) in glossary
  \newignoredglossary{hidden}
  \newcommand*{\hidegacronym}[1]{\glsmoveentry{Ü#1}{hidden}}
  
  % spell out acronyms at beginning of chapters and sections
  \newcommand*{\resetacronyms}{%
    \forglsentries{\mysymbolname}{%
      \ifthenelse{\equal{\glsmyisacronym{\mysymbolname}}{true}}{
        \glsreset{\mysymbolname}%
      }{}%
    }%
  }
  
  \addtokomafont{chapter}{\resetacronyms}
  \addtokomafont{section}{\resetacronyms}
  
  % warn for every symbol that has not been used
  \AtEndDocument{%
    \forglsentries{\mysymbolname}{%
      \ifthenelse{\equal{\glsmyused{\mysymbolname}}{false}}{
        \GenericWarning{}{%
          LaTeX Warning: Symbol ``\mysymbolname'' defined, but unused%
        }%
      }{}%
    }%
  }
  
  % generate table of symbols and acronyms
  \makeglossaries
}

% add debug info about definitions of glossary entries
\iftoggle{showGlossaryDefinitionsMode}{
  \let\oldnewgsymbol\newgsymbol
  \renewcommand*{\newgsymbol}[3]{%
    \ifx\@onlypreamble\@notprerr%
      \textcolor{C1}{Defining ``#2'' as ``#3''. }%
    \fi%
    \oldnewgsymbol{#1}{#2}{#3}%
  }
}{}
