% ======================================================================
% Theorems
% ======================================================================

% define mdframed style for theorems
\mdfdefinestyle{thmmdfstyle}{
  innerleftmargin=13pt,
  innerrightmargin=7pt,
  innertopmargin=5pt,
  innerbottommargin=6pt,
  linewidth=1pt,
  linecolor=mittelblau,
  backgroundcolor=mittelblau!10,
  % the documentation of mdframed lies about the default values of
  % splittopskip and splitbottomskip
  splittopskip=19pt,
  splitbottomskip=5pt,
  % thick line on left side
  extra={
    \draw[mittelblau,line width=5pt]
    ($(O)+(2.5pt,0pt)$) -- ($(O|-P)+(2.5pt,0pt)$);
  },
}

% "torn page" effect at bottom of the box before page break
\newcounter{mathseed}
\setcounter{mathseed}{3}
\pgfmathsetseed{\arabic{mathseed}}

\tikzset{
  every picture/.append style={
    execute at begin picture={
      \ifboolexpr{
        bool{mdffirstframe} or bool{mdfmiddleframe} or bool{mdflastframe}
      }{
        \mdf@innertikzcalc
        \coordinate (O) at (\mdf@Ox,\mdf@Oy);
        \coordinate (P) at (\mdf@Px,\mdf@Py);
        \ifboolexpr{bool{mdffirstframe} or bool{mdfmiddleframe}}{
          \clip ($(O)+(-0.5pt,2pt)$)
          \pgfextra{
            \addtocounter{mathseed}{1}
            \pgfmathsetseed{\arabic{mathseed}}
          }
          decorate[
            decoration={my random steps,segment length=2pt,amplitude=2pt}
          ]{-- ($(P|-O)+(0.5pt,2pt)$)}
          -- ($(P)+(0.5pt,0.5pt)$)
          -- ($(O|-P)+(-0.5pt,0.5pt)$)
          -- cycle;
        }{}
        \ifboolexpr{bool{mdflastframe} or bool{mdfmiddleframe}}{
          \clip ($(O)+(-0.5pt,-0.5pt)$)
          -- ($(O|-P)+(-0.5pt,-2pt)$)
          \pgfextra{\pgfmathsetseed{\arabic{mathseed}}}
          decorate[
            decoration={my random steps,segment length=2pt,amplitude=2pt}
          ]{-- ($(P)+(0.5pt,-2pt)$)}
          -- ($(P|-O)+(0.5pt,-0.5pt)$)
          -- cycle;
        }{}
      }{}
    }
  }
}

% for the "torn page" effect,
% derived from the "random steps" decoration, the difference is that
% it only add random pertubations in y-direction, not in x-direction
\pgfdeclaredecoration{my random steps}{start}{
  \state{start}[
    width=+0pt,next state=step,
    persistent precomputation=\pgfdecoratepathhascornerstrue
  ]{}
  \state{step}[
    auto end on length=1.5\pgfdecorationsegmentlength,
    auto corner on length=1.5\pgfdecorationsegmentlength,
    width=+\pgfdecorationsegmentlength
  ]{
    \pgfpathlineto{
      \pgfpointadd{
        \pgfpoint{\pgfdecorationsegmentlength}{0pt}
      }{
        \pgfpoint{0pt}{rand*\pgfdecorationsegmentamplitude}
      }
    }
  }
  \state{final}{}
}

% patch mdframed such that it doesn't violate \flushbottom
% (mdframed produces ragged page bottoms without this)
\xpatchcmd{\mdf@put@frame@i}{\hrule \@height\z@ \@width\hsize\vfill}{}{}{}
\xpatchcmd{\mdf@put@frame@i}{\hrule \@height\z@ \@width\hsize\vfill}{}{}{}
\xpatchcmd{\mdf@put@frame@i}{\hrule \@height\z@ \@width\hsize\vfill}{}{}{}

% ignore "bad break" warnings by mdframed, they don't seem to make sense
% (they claim that the box after the page break would be empty,
% even if it isn't)
\WarningFilter{mdframed}{You got a bad break}

% define theorem styles (format theorem "head" with sans-serif caps)
\declaretheoremstyle[
  headformat={\formatcaption{\NAME{} \NUMBER}\hspace{0.7em}\NOTE\\},
  notefont={\normalfont\normalsize},
  headpunct={},
]{mythmdefstyle}

% same as before, but italic body font
\declaretheoremstyle[
  headformat={\formatcaption{\NAME{} \NUMBER}\hspace{0.7em}\NOTE\\},
  notefont={\normalfont\normalsize},
  headpunct={},
  bodyfont={\itshape},
]{mythmplainstyle}

% same as before, but no note and line break after head
\declaretheoremstyle[
  headformat={\formatcaption{\NAME{} \NUMBER}\hspace{0.7em}},
  notefont={\normalfont\normalsize},
  headpunct={},
  bodyfont={\itshape},
]{mythmshortplainstyle}

% theorem environments
\declaretheorem[
  name={Definition},
  numberwithin=chapter,
  mdframed={style=thmmdfstyle},
  style=mythmdefstyle,
]{definition}
\declaretheorem[
  name={Theorem},
  numberlike=definition,
  mdframed={style=thmmdfstyle},
  style=mythmplainstyle,
]{theorem}
\declaretheorem[
  name={Proposition},
  numberlike=definition,
  mdframed={style=thmmdfstyle},
  style=mythmplainstyle,
]{proposition}
\declaretheorem[
  name={Lemma},
  numberlike=definition,
  mdframed={style=thmmdfstyle},
  style=mythmplainstyle,
]{lemma}
\declaretheorem[
  name={Lemma},
  numberlike=definition,
  mdframed={style=thmmdfstyle},
  style=mythmshortplainstyle,
]{shortlemma}
\declaretheorem[
  name={Corollary},
  numberlike=definition,
  mdframed={style=thmmdfstyle},
  style=mythmplainstyle,
]{corollary}
\declaretheorem[
  name={Corollary},
  numberlike=definition,
  mdframed={style=thmmdfstyle},
  style=mythmshortplainstyle,
]{shortcorollary}
