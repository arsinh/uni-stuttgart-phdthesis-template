\RequirePackage{luatex85}   % This issue will be resolved in TexLive 2018
\documentclass{standalone}
\usepackage{regexpatch, shellesc}
\usepackage{tikz}
\usetikzlibrary{shapes}
\usepackage{python}

% Example how to draw a regular sparse grid in 2D without boundary points using
% python package and TikZ.
% Note that comments in python part have to be in python syntax.
% Compile with --enable-write18

\begin{document}
  \begin{python}
lmax = 5

# calculate points in 1D
pts = []
for l in range(1, lmax + 1):
    lvlPts = []
    for x in range(2 ** l):
        if x % 2 == 1:
            lvlPts.append(x / float(2**l))
    pts.append(lvlPts)


# calculate tensor product (by hand because of max level sum)
sgpts = []
for x in range(lmax):
    for y in range(lmax):
        # note that index is shifted by 1 per dimension.
        # without shift this line would read (x + y) < (lmax + 1)
        if x + y < lmax:
            for a in pts[x]:
                for b in pts[y]:
                    sgpts.append([a, b])

# setup tikz env (scale by lmax to avoid bullets being too close together)
print(r"\begin{tikzpicture}[scale=" + repr(lmax) + "]")

# draw bounding box
print(r"\draw (0, 0) rectangle (1, 1);")

# draw sparse grid nodes
for p in sgpts:
    print(r"\node at (" + repr(p[0]) + ", " + repr(p[1]) + ") {\\textbullet};")

# close tikz env
print(r"\end{tikzpicture}")
  \end{python}
\end{document}
